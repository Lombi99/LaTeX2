\documentclass[10pt,a4paper]{article}
\usepackage[utf8]{inputenc}
\usepackage[italian]{babel}
\usepackage[a4paper]{geometry}
\usepackage{amsmath}
\usepackage{amsthm}
\usepackage{dsfont}
\usepackage{xfrac}
\usepackage{amsfonts}
\usepackage{amssymb}
\usepackage{graphicx}
\usepackage{braket}
\usepackage{mathtools}
\usepackage{booktabs}
\usepackage{hyperref}
\usepackage{enumerate}
\DeclarePairedDelimiterX{\norm}[1]{\lVert}{\rVert}{#1}
\theoremstyle{plain}
\newtheorem{definizione}[subsection]{Definizione}
\theoremstyle{definition}
\newtheorem{teorema}[subsection]{Teorema}
\newtheorem{dimostrazione}{Dimostrazione}
\newtheorem{corollario}[subsection]{Corollario}
\newtheorem{osservazione}[subsection]{Osservazione}
\newtheorem{proposizione}[subsection]{Proposizione}
\newtheorem{esempio}[subsection]{Esempio}
%%%%%%%INSERIAMO QUI I NUOVI COMANDI %%%%%%%%%%%%%%%%%%%%%%%%%%%%%%%%%%%%%%%%%%%%
\DeclarePairedDelimiterX{\abla}[1]{abla}{abla}{#1}

%%%%%%%QUI VANNO INSERITI IL TITOLO E GLI AUTORI%%%%%%%%%%%%%%%%%%%%%%%%%%%%
\author{Andrea Zanin, Nicola Lombardi}
\title{\textbf{Autovettori, Autovalori e Metodo del Cambiamento di Base}}
\date{8 Giugno 2017}

\begin{document}
\maketitle
\section{Autovettori e autovalori}
\begin{definizione}
	Un vettore ${v} \in \mathbb{R}^2 -\{{0}\}$ è detto autovettore della matrice M se $M{v}\parallel{v}$, cioè: \\
	\[\exists \lambda \in \mathbb{R} \vert M{v}=\lambda{v}\]
	$\lambda$ è detto autovalore di M.
\end{definizione}
\begin{teorema}$\lambda$ è autovalore di $M=\begin{pmatrix}
	a & c \\ b & d
	\end{pmatrix} \Leftrightarrow \det{\begin{pmatrix}
		a-\lambda & c \\ b & d-\lambda
		\end{pmatrix}}=0$. \\
	
\proof
\[\]
Siano:

$a,b,c,d,\lambda\in\mathbb{R}$ 

${v}=
\begin{pmatrix}
v_1 \\ v_2
\end{pmatrix}$ autovettore di M riferito a $\lambda\quad\quad v_1,v_2\in\mathbb{R}$ 

Allora, applicando le proprietà di somma e prodotto di vettori e prodotto matrice vettore, otteniamo che:
\begin{align*}
	\begin{pmatrix}
	a & c \\ b & d
	\end{pmatrix}
	\cdot
	\begin{pmatrix}
	v_1 \\ v_2
	\end{pmatrix}
	&=
	\lambda
	\begin{pmatrix}
		v_1 \\ v_2
	\end{pmatrix}
	\\
	\begin{pmatrix}
	av_1 + cv_2 \\ bv_1 + dv_2
	\end{pmatrix}
	&=
	\begin{pmatrix}
	\lambda v_1 \\ \lambda v_2
	\end{pmatrix}
	\\
	\begin{pmatrix}
	(a-\lambda)v_1 + cv_2 \\ bv_1 + (d-\lambda)v_2
	\end{pmatrix}
	&=
	\begin{pmatrix}
		0  \\ 0
	\end{pmatrix}
	\\
	\begin{pmatrix}
		a-\lambda & c \\ b & d-\lambda
	\end{pmatrix}
	\begin{pmatrix}
		v_1 \\ v_2
	\end{pmatrix}
	&=
	\begin{pmatrix}
		0 \\ 0
	\end{pmatrix}
\end{align*}
Se esiste un autovettore la funzione associata alla matrice $\begin{pmatrix}
a-\lambda & c \\ b & d-\lambda
\end{pmatrix}$ non può essere iniettiva in quanto l'autovettore e il vettore $\begin{pmatrix}
	0 \\ 0
\end{pmatrix}$ hanno entrambi come immagine $\begin{pmatrix}
	0 \\ 0
\end{pmatrix}$, quindi, dato che la funzione associata ad una matrice non è iniettiva $\Leftrightarrow \det{M}=0$, 
\[
\lambda \text{ è autovalore di }M = \begin{pmatrix}
	a & c \\ b & d
\end{pmatrix} \Leftrightarrow \det{\begin{pmatrix}
	a-\lambda & c \\ b & d-\lambda
\end{pmatrix}}=0\text{.}
\]

\qed
\end{teorema}
\newpage
\begin{osservazione}
	Tutti i vettori paralleli ad un autovettore sono anch'essi autovettori riferiti allo stesso autovalore.
\end{osservazione}
\proof
	\[\]
	 Siano:
	
	$M$ matrice
	
	$\lambda$ autovalore di M
	
	${u}$ autovettore riferito a $\lambda$
	
	Allora:
	\[
	M({u})=\lambda{u}
	\]
	Moltiplico entrambi i membri per $k\in\mathbb{R}$
	\[
	k(M({u}))=k(\lambda{u})
	\]
	Dato che la funzione associata alla matrice è lineare, posso scrivere che
	\[
	M(k{u})=\lambda k{u}
	\]
	Chiamando $k{u}={v}$
	\[
	M({v})=\lambda{v}
	\]
	Di conseguenza anche tutti i vettori ${v}$, paralleli a ${u}$ per costruzione, sono autovettori di $M$ riferiti a $\lambda$.
	 
	\qed

\begin{definizione}
	Una matrice si dice simmetrica se è del tipo $M=\begin{pmatrix}
		a & b \\ b & d
	\end{pmatrix}$.
\end{definizione}
\begin{teorema} \label{delta}
	Ogni matrice simmetrica ha almeno un autovalore. \\
\proof
\[\]
Cerco gli autovalori di una generica matrice simmetrica
	\begin{align*}
		0&=\det{\begin{pmatrix}
			a-\lambda & b \\ b & d-\lambda
			\end{pmatrix}}
		\\
		&=(a-\lambda)(d-\lambda)-b^2\\
		&=\lambda^2 -\lambda(a+b) +ad -b^2
	\end{align*}
	Calcolando il discriminante, otteniamo che:
	\begin{align*}
		\Delta&=(a+d)^2-4(ad-b^2) \\
		&=a^2+d^2+2ad-4ad+4b^2\\
		&=(a-d)^2 + 4b^2\\
	\end{align*}
	Di conseguenza, essendo una somma tra due quadrati:
	\[
		(a-d)^2 + 4b^2 \ge 0 \quad \forall \; a,b,d\in\mathbb{R}
	\]
	Essendo il $\Delta$ non negativo indifferentemente dalla scelta dei parametri, esiste sempre almeno un autovalore di una matrice simmetrica.
	\qed
\end{teorema}
\newpage
\begin{teorema}
	Una matrice simmetrica ha sempre 2 autovettori ortogonali fra loro. 
	\proof
	\[\]
	Iniziamo dimostrando il caso in cui $b=0$, quindi in cui $M=\begin{pmatrix}
	a & 0 \\ 0 & d
	\end{pmatrix}$\\
	Applicando la matrice ai vettori $\begin{pmatrix}
	1 \\ 0 
	\end{pmatrix}$ e $\begin{pmatrix}
	0 \\ 1 
	\end{pmatrix},$ otteniamo che:\\
	\begin{align*}
		M\begin{pmatrix}
			1 \\ 0
		\end{pmatrix}&=\begin{pmatrix}
			a \\ 0
		\end{pmatrix}
		\\
		&=a\begin{pmatrix}
		1 \\ 0
		\end{pmatrix}\\
		M\begin{pmatrix}
		0 \\ 1
		\end{pmatrix}&=\begin{pmatrix}
		0 \\ d
		\end{pmatrix}
		\\
		&=d\begin{pmatrix}
		0 \\ 1
		\end{pmatrix}
	\end{align*}
	$\begin{pmatrix}
		1 \\ 0
	\end{pmatrix}$ e $\begin{pmatrix}
		0 \\ 1
	\end{pmatrix}$ sono quindi due autovettori ortogonali tra loro.\\
	\[\]
Supponendo poi che $b\ne 0$, per il teorema \ref{delta}, $\Delta > 0$, di conseguenza $\lambda_1 \ne \lambda_2$. Sia $\mathbf{u}$ un qualsiasi autovettore relativo a $\lambda_1$ e sia $\mathbf{v}$ un qualsiasi autovettore relativo a $\lambda_2$
\begin{align*}
M\mathbf{u}&=\lambda_1 \mathbf{u} \\
M\mathbf{v}&=\lambda_2 \mathbf{v} 
\end{align*}
È un fatto noto che:
\[
M(\mathbf{u})\mathbf{v}=M(\mathbf{v})\mathbf{u}
\]
Allora, sostituendo dalle prime due equazione $M(\mathbf{u})$ e $M(\mathbf{v})$, si ottiene:
\begin{align*}
\lambda_1 \mathbf{u} \mathbf{v} &= \lambda_2 \mathbf{v}\mathbf{u} \\
(\lambda_1 - \lambda_2)\mathbf{u}\mathbf{v} &=0
\end{align*}
Avendo dimostrato che 
$\lambda_1 - \lambda_2 \ne 0$, è necessario che $\mathbf{u}\cdot\mathbf{v}=0$, quindi $\mathbf{u}$ e $\mathbf{v}$ sono ortogonali.\\ \qed
\end{teorema}
\section{Metodo del cambiamento di base}
 È un metodo per risolvere un sistema di equazioni lineari. \\
Partiamo dal sistema
\[
\begin{cases*}
ap_1 + bp_2 = c \\
dp_1 + ep_2 = f
\end{cases*}
\]
e lo trasformiamo in forma matriciale
\[
\begin{pmatrix}
a & b \\
d & e
\end{pmatrix}
\begin{pmatrix}
p_1 \\
p_2
\end{pmatrix}
= \begin{pmatrix}
c \\
f
\end{pmatrix}
\]
Troviamo gli autovalori $\lambda_1$ e $\lambda_2$ della matrice $\begin{pmatrix}
a & b \\
d & e
\end{pmatrix}$ e un autovalore riferito ad ognuno dei 2 autovalori, chiamamo quest'ultimi $\mathbf{u}$ e $\mathbf{v}$. \\
Successivamente troviamo le coordinate $q_1$ e $q_2$ del vettore $\begin{pmatrix}
c \\
f
\end{pmatrix}$ nel sistema di riferimento determinato dai 2 autovettori risolvendo la seguente equazione
\[
\begin{pmatrix}
c \\
f
\end{pmatrix}=q_1\mathbf{u} + q_2\mathbf{v}
\]
Possiamo quindi ottenere le coordinate $p_1'$ e $p_2'$, corrispondenti di $p_1$ e $p_2$ nel sistema di riferimento determinato dagli autovettori, utilizzando la formula 
\[
\begin{pmatrix}
p_1 \\
p_2
\end{pmatrix}=p_1'\mathbf{u}+p_2'\mathbf{v}
\]
Con queste coordinate possiamo riscrivere la precedente rappresentazione matriciale del sistema, ma nel nuovo sistema di riferimento
\begin{align*}
M(p_1'\mathbf{u}+p_2'\mathbf{v})&=q_1\mathbf{u}+q_2\mathbf{v}\\
p_1'M(\mathbf{u})+p_2'M(\mathbf{v})&=q_1\mathbf{u}+q_2\mathbf{v}
\end{align*}
Dato che $\mathbf{u}$ e $\mathbf{v}$ sono autovettori
\begin{align*}
M\mathbf{u}&=\lambda_1 \mathbf{u} \\
M\mathbf{v}&=\lambda_2 \mathbf{v}
\end{align*}
Posso quindi sostiturli nell'equazione
\[
p_1'\lambda_1\mathbf{u}+p_2'\lambda_2\mathbf{v}=q_1\mathbf{u}+q_2\mathbf{v}
\]
Dato che le coordinate di un vettore rispetto alle basi del sistema cartesiano sono uniche, l'equazione precedente è equivalente al seguente sistema
\[
\begin{cases*}
\lambda_1p_1'=q_1 \\
\lambda_2p_2'=q_2	
\end{cases*}
\]
Risolvendo il sistema troviamo che $p_1'=\frac{q_1}{\lambda_1}$ e $p_2'=\frac{q_2}{\lambda_2}$\\
Possiamo quindi trovare le soluzioni $p_1$ e $p_2$ del sistema iniziale riportando $p_1'$ e $p_2'$ nel sistema di riferimento originale
\[
\begin{pmatrix}
p_1 \\
p_2
\end{pmatrix}
=
p_1'\mathbf{u}+p_2'\mathbf{v}\text{.}
\]
\end{document}